\documentclass[12pt,twoside]{rif}

\pagestyle{myheadings}
\usepackage[
left=2.54cm,
right=2.54cm,
top=2.54cm,
bottom=2.54cm]
{geometry}
\usepackage{hyperref}
\usepackage{natbib}
\usepackage{subfigure}
\hypersetup{
	urlcolor=blue, 
	colorlinks=true, 
	citecolor=blue
}

\usepackage{lipsum}

\title{\textbf{Dilatación del tiempo gravitacional}}

\author[1]{{\small Carlos Andrés García Suarez}} 
\author[1]{{\small David Brandon Zevallos Garay}}
\author[1]{{\small Luis Fernando Ubillus Benites}}
%\author[3]{Autor3}
\affil[1]{{ \small Facultad de Ciencias Naturales y Matemática, Universidad
		Nacional Federico Villarreal. El Agustino 15003. Lima-Perú.}}
%\affil[2,3]{Afiliacion2}
%\date{\normalsize Recibido: xxxx Aceptado: xxxx Publicado: xxxx\\
%Todos los derechos reservados-SEF \copyright{} 2012}
\date{}

\begin{document}
	\maketitle
	
	\begin{res}
		\begin{center}
			\textbf{Resumen} \\
		\end{center}
		\lipsum[2]
		
		\par
		\smallskip
		\clav{sdjssasdfsdf, asdfsdf, asdfsdafsd}
	\end{res}
	\begin{center}
		\title{\textbf{Time gravitational dilation}}
	\end{center}
	
	\begin{abst}
		\begin{center}
			\textbf{Abstract} \\
		\end{center}
		\lipsum[2]
		
		\par 
		\smallskip
		\key{sdjssasdfsdf, asdfsdf, asdfsdafsd}
	\end{abst}

	
	
	\newpage
	
	\tableofcontents
	
	\section{ Introducción} 
	
	\section{Marco Teórico}
		\subsection{Masa gravitatoria y masa inercial}
			
		
		\subsection{Principio de equivalencia}
			
	\section{Conclusiones}
	
	\nocite{*}
	\bibliographystyle{apa}
	\bibliography{biblio}
	

\end{document}